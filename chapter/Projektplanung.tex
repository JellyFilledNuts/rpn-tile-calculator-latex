%!TEX root = ../Thesis.tex
\section{Projektplanung}

\subsection{Beschreibung des Funktionsumfangs}

\subsection{Projektablaufplan}

\subsection{Planung der Software}

\subsubsection{Planung des Mockups}

\subsubsection{Planung der Datenstrukturen und Schnittstellen}

\para{Nutzung von Stack für Notation [Schwenke]}

Der Taschenrechner soll als Eingabelogik für die Anwendung von Operationen die umgekehrte polnische Notation verwenden. Hierbei werden immer zunächst die Operanden und im Anschluss daran die darauf auszuführenden Operatoren angegeben. Dieser Ansatz ermöglicht eine stapelbasierte Abarbeitung. 

Stacks werden, wie von den meisten Programmiersprachen, auch in Java in der Standardbibliothek unterstützt. Mit dabei sind Methoden wie \code{push} (für das Ablegen eines Objekts auf dem Stapel), \code{pop} (für das Entfernen und die Wiedergabe eines Objekts auf dem Stapel), \code{peek} (für die Wiedergabe ohne Entfernen eines Objekts auf dem Stapel) und \code{empty} (für das Leeren des Stapels). 

Jedoch müssen hierbei die besonderen Anforderungen des Taschenrechners beachtet werden. Operanden können von gänzlich unterschiedlichem Typus sein, zum Beispiel eine einfache Dezimalzahl oder auch ein Tupel, und viele Operationen benötigen mehr als die ersten (maximal zwei) Operanden auf dem Stack. Möchte man Elemente vom Stapel entfernen, kann man \code{pop} mehrmals aufrufen. Aufwändiger hingegen wird es bei \code{peek}. Möchte man mehrere Elemente vom Stapel einsehen ohne diese zu entfernen, muss man bei der Arbeit mit dem vorhandenen Stack einen weiteren bereithalten, nur um zwischengespeicherte Elemente lagern zu können. Anders ist es nicht möglich \code{peek} auf mehrere Elemente gleichzeitig anzuwenden. Gerade das ist aber bei der App notwendig. Weitere Methoden, die bei der umgekehrten polnischen Notation oft benötigt werden, aber nicht implementiert sind, sind \code{reverse} (für die Vertauschung der ersten zwei Elemente auf dem Stack, was wichtig für nicht-kommutative Operationen ist), \code{rollUp} (das unterste Elemente wird an den ersten Platz geschoben, das erste Element an den zweiten Platz usw.) und \code{rollDown} (das unterste Elemente wird an den ersten Platz geschoben, das erste Element an den zweiten Platz usw.).

Aufgrund dessen soll für dieses Projekt ein eigener Stapel implementiert werden. Dieser soll die zuvor genannten Funktionen mit unterschiedlichen Parametertypen unterstützen. Dabei ist darauf zu achten, dass die Programmierung generisch erfolgt und das Stack nicht nur alle Typen von Operanden unterstützt, sondern auch für gänzlich andere Klassenbäume in der App verwendet werden kann.

\para{Ansatz der Kalkulationsorchestrierung [Schwenke]}

Die App soll den Umgang mit unterschiedlichen Operanden-Typen beherrschen. Die Addition zweier Matrizen funktioniert anders als die Addition von zwei einfachen Dezimalzahlen. Java verfügt nativ weder über die entsprechenden Operanden noch über die Methoden für die Kalkulation. Auch die ausgewählte Bibliothek ist nicht ohne weiteres in der Lage Operationen auf alle Kombinationen von Operanden im folgenden Format einheitlich anzuwenden:

\begin{figure}[bht]
	\begin{lstlisting}[caption=Konzept für Nutzung generischer Schnittstelle, label=list:konzept-fuer-nutzung-generischer-schnittstelle]
Operation.mit(matrixOperand, dezimalOperand, dezimalOperand)
	\end{lstlisting}    
\end{figure}

Einheitlichkeit ist notwendig, damit im Frontend der Applikation keine Logik vorhanden sein muss, die entscheidet wie genau (auf Basis der Operanden-Typen) eine Operation umgesetzt wird. Deswegen muss eine einfache Schnittstelle entwickelt werden, die für den Nutzer nur zwei Drehschrauben bereitstellt. Dies ist zunächst die Auswahl der gewünschten Operation. Das kann z.B. das Symbol \code{+} als übliches Zeichen für Addition sein. Anschließend wird eine Reihe von Operanden übergeben. Dieser Aufruf sollte schließlich das Ergebnis in Form eines Operanden zurückgeben. Im Fall der Addition einer Matrix mit einer rationalen Zahl wäre dies wiederrum eine Matrix. Die korrekte Kalkulation soll also dynamisch bestimmt werden. Wichtig zu klären ist hier auch das Verhalten im Falle eines Fehlschlags. Nicht alle Kombinationen von Operanden können unterstützt werden. Die Verwendung von \textit{Optionals} (ein \code{Optional} ist ein Objekt, das man sich als Datenbehälter vorstellen kann, der entweder einen Wert enthält oder leer – aber nicht \code{null} sein kann) bietet sich hier zwar an, wird jedoch von Java in der verwendeten Android API-Version nicht unterstützt. Deswegen ist hier geplant sogenannte \textit{checked Exceptions} zu verwenden. Diese müssen bei der Verwendung explizit aufgefangen und weiterverarbeitet werden. Die Abbildung einer Operanden-Kombination auf die entsprechende konkrete Kalkulationsmethode muss dementsprechend zur Laufzeit des Programms erfolgen. Ein solches Mapping ist in Java nur mithilfe des Reflection-Pakets möglich. Reflektion ermöglicht den Einblick in ein Objekt (neben der Nutzung des Punkt-Operators) in eine Klasse. Zum Beispiel kann man eine Methode anhand einer Kombination von Parametertypen finden und aufrufen. Es ist geplant diesen Ansatz für die Orchestrierung der Kalkulationen in der App zu verwenden. Auch ist es nicht notwendig nur eine vordefinierte Anzahl an Argumente anzunehmen. So kann es sinnvoll sein, dass eine Methode zur Erstellung eines Tupels eine beliebige Anzahl an Operanden annimmt. Auch das lässt sich mit Reflektion umsetzen.

Der große Vorteil dabei ist, dass nirgendwo explizit in einer Abfrage entschieden werden muss, welche Kombination von Operanden an welche Methode weitergeleitet werden soll. Die Zuordnung erfolgt rein über die Deklaration der Parametertypen in der Methode selbst. Das macht das Ändern und Erweitern der Rechenfunktionalitäten einfach. Es muss lediglich die entsprechende Klasse herausgesucht und eine Methode im korrekten Format hinzugefügt werden. 

Zu entscheiden ist ebenfalls, ob das Gros der Rechenmethoden innerhalb der jeweiligen Operanden-Klassen oder dedizierten Klassen für die Kalkulation angesiedelt sind. Die erste Option hat neben der stärkeren Objektorientierung den Vorteil, dass immer klar ist, dass eine Methode mit den übergebenen Argumenten auf dem jeweiligen Objekt ausgeführt wird. Andererseits erhöht dies die Komplexität der Operanden-Klassen deutlich. Unterstützt man wie geplant 5 bis 7 dedizierte Typen von Operanden und 10 Kalkulationsarten, muss jede Klasse potenziell dutzende Methoden für die Rechnung enthalten. Die andere, und bevorzugte Option, ist die Auslagerung der Kalkulationsmethoden in eigenständige Klassen. Dies reduziert zwar nicht die Anzahl benötigter Methoden, isoliert die Rechenlogik jedoch in Klassen. Innerhalb dieser Klassen wird prozedural programmiert.  Eine typische Charakteristik von Objekten und deren Methoden ist \textit{Mutability}. Eine Methode bekommt ein Objekt und kann dieses verändern. Dies kann Testen unter Umständen aufwändiger gestalten. Durch Isolierung der Rechnungen in eigenen Klassen kann hingegen sichergestellt werden, dass jede Methode \textit{immutable}, also unveränderlich, ist. Das macht das Schreiben von Tests einfach. In Java kann Immutability durch die Verwendung von Annotationen sichergestellt werden. 

\subsubsection{Planung der Activities und Layouts}

\subsubsection{Planung der Navigation zwischen den Activities}

\subsection{Geplante Aufgabenverteilung im Team (tabellarisch)}